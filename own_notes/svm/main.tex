\documentclass[letterpaper,11pt]{article}
\usepackage{xeCJK}
\usepackage{bm}
\usepackage{geometry}
\usepackage{amssymb}
\usepackage{amsmath}
\usepackage[cal=boondoxo, scr=dutchcal]{mathalfa}
\usepackage{multirow}
\usepackage[unicode, CJKbookmarks=true]{hyperref}

\numberwithin{equation}{section}

\geometry{a4paper,left=2cm,right=2cm,top=2cm,bottom=2cm}

\usepackage{setspace}
\setstretch{1.1}
\setlength{\parskip}{0.1\baselineskip}

\usepackage{graphicx}
\usepackage{caption}

\begin{document}

\title{支持向量机}
\author{张驰}
\maketitle

\begin{equation}
    h_{{\vec{w}},b}(\vec{x})={\rm sign}(\vec{w} \cdot \vec{x}+b)
\end{equation}

\begin{equation}
    \hat{\gamma}^{(i)}=y^{(i)}(\vec{w} \cdot \vec{x}^{(i)}+b)
\end{equation}

对$\vec{w}$与$b$缩放不会改变$h_{{\vec{w}},b}(\vec{x})$的实际结果。

\begin{equation}
    \hat{\gamma} = \mathop{min} \limits_{i=1,\dots,n} \frac{\hat{\gamma}^{(i)}}{\left \| \vec{w} \right \|}
\end{equation}

\begin{equation}
    \begin{aligned}
        max_{\vec{w},b} \quad & \gamma                                                                 \\
        s.t. \quad            & y^{(i)}(\vec{w} \cdot \vec{x}^{(i)} + b) \ge \gamma, \quad i=1,\dots,n
    \end{aligned}
\end{equation}

由于$\vec{w}$与$b$可以任意缩放,令$\gamma=1$。

\begin{equation}
    \begin{aligned}
        min_{\vec{w},b} \quad & \frac{1}{2} {\left \| \vec{w} \right \|}^2                        \\
        s.t. \quad            & y^{(i)}(\vec{w} \cdot \vec{x}^{(i)} + b) \ge 1, \quad i=1,\dots,n
    \end{aligned}
\end{equation}

拉格朗日函数,其中 $\alpha_i>=0$。

\begin{equation}
    \mathcal{L}(\vec{w},b,\vec{\alpha})=\frac{1}{2} {\left \| \vec{w} \right \|}^2-\sum_{n}^{i=1}\alpha_i \left[ y^{(i)}(\vec{w} \cdot \vec{x}^{(i)} + b)-1 \right]
    \label{eq:lagrange}
\end{equation}

这里跳一步,直接转化为对偶问题:
\begin{equation}
    \mathop{max}_{\vec{\alpha}}\mathop{min}_{\vec{w},b}\mathcal{L}(\vec{w},b,\vec{\alpha})
\end{equation}

\begin{equation}
    \nabla_{\vec{w}}\mathcal{L}(\vec{w},b,\vec{\alpha})=\vec{w}-\sum_{i=1}^{n}\alpha_iy^{(i)}\vec{x}^{(i)}=0
    \label{eq:der_w}
\end{equation}

\begin{equation}
    \frac{\partial}{\partial b}\mathcal{L}(\vec{w},b,\vec{\alpha})=\sum_{i=1}^{n}\alpha_i y^{(i)}=0
    \label{eq:der_b}
\end{equation}

将等式\ref{eq:der_w}与等式\ref{eq:der_b}代入等式\ref{eq:lagrange}得到:

\begin{equation}
    \mathcal{L}(\vec{\alpha})=\sum_{i=1}^{n}\alpha_i - \frac{1}{2}\sum_{i,j=1}^{n}y^{(i)}y^{(j)}\alpha_i \alpha_j \langle \vec{x}^{(i)} , \vec{x}^{(j)} \rangle
\end{equation}

\begin{equation}
    \begin{aligned}
        max_{\vec{\alpha}} \quad & \mathcal{L}(\vec{\alpha})=\sum_{i=1}^{n}\alpha_i - \frac{1}{2}\sum_{i,j=1}^{n}y^{(i)}y^{(j)}\alpha_i \alpha_j \langle \vec{x}^{(i)} , \vec{x}^{(j)} \rangle \\
        s.t. \quad               & a_i \ge 0, i=1,\dots,n                                                                                                                                      \\
                                 & \sum_{i=1}^{n}\alpha_iy^{(i)}=0
    \end{aligned}
\end{equation}

\begin{equation}
    \begin{aligned}
        \vec{w}^{*} & =\sum_{i=1}^{n}\alpha_iy^{(i)}\vec{x}^{(i)}                                                               \\
        b^*         & =- \frac{1}{2} (max_{i:y^{(i)}=-1}\vec{w}^{*} \cdot x^{(i)} + min_{i:y^{(i)}=1}\vec{w}^{*} \cdot x^{(i)})
    \end{aligned}
\end{equation}

\begin{equation}
    \begin{aligned}
        min_{\vec{w},b} \quad & \frac{1}{2} {\left \| \vec{w} \right \|}^2 + C \sum_{i=1}^{n} \xi_i \\
        s.t. \quad            & y^{(i)}(\vec{w} \cdot \vec{x}^{(i)} + b) \ge 1, \quad i=1,\dots,n   \\
                              & \xi_i \ge 0, \quad i=1,\dots,n
    \end{aligned}
\end{equation}

\begin{equation}
    \mathcal{L}(\vec{w},b,\vec{\xi},\vec{\alpha},\vec{r})=\frac{1}{2} {\left \| \vec{w} \right \|}^2 + C \sum_{i=1}^{n} \xi_i -\sum_{n}^{i=1}\alpha_i \left[ y^{(i)}(\vec{w} \cdot \vec{x}^{(i)} + b)-1 \right] - \sum_{i=1}^{n} r_i \xi_i
    \label{eq:lagrange-soft-margin}
\end{equation}

\end{document}