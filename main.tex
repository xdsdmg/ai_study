\documentclass[letterpaper,11pt]{article}
\usepackage{CJKutf8}
\usepackage{bm}
\usepackage{geometry}
\usepackage{amssymb}
\usepackage{amsmath}
\usepackage[cal=boondoxo]{mathalfa}

\numberwithin{equation}{section}

\geometry{a4paper,left=2cm,right=2cm,top=1cm,bottom=1cm}

\usepackage{setspace}
\setstretch{1.1}
\setlength{\parskip}{0.1\baselineskip}

\begin{document}
\begin{CJK}{UTF8}{gbsn}

  \section{分类与逻辑回归}

  \subsection{多分类问题}

  考虑这样一个分类问题,响应值$y$可以为指定的$k$个值中的任意一个,即$y\in\{1,2,\ldots,k\}$。举个例子,我们可能想要将邮件划分为三种类型,比如垃圾邮件、个人邮件及工作邮件,而不是只划分为垃圾邮件与非垃圾邮件(这是一个二分类问题)。标签或响应值仍然是离散的,但可以取两个以上的值。因此我们将使用多项式分布对其进行建模。

  在这种情况下,$p(y | x;\theta)$是基于$k$个可能的离散值的分布,因此是一个多项式分布。对于包含$k$个值的多项式分布,$\phi_1,\ldots,\phi_k$表示每一种可能的概率,必须满足约束$\sum^k_{i=1}\phi_i=1$。我们将设计一个参数化模型,在给出输入$x$的前提下,输出满足这个约束的$\phi_1,\ldots,\phi_k$。

  我们引入$k$组参数$\theta_1,\ldots,\theta_k$,每一组参数都是空间$\mathbb{R}^d$中的一个向量。根据直觉,我们应该可以使用$\theta_1^Tx,\ldots,\theta_k^Tx$来表示$\phi_1,\ldots,\phi_k$,即概率$P(y=1 | x;\theta),\ldots,P(y=k | x;\theta)$。然而,采用这种直接的办法有两个问题,首先,$\theta_j^Tx$不一定在$[0,1]$内,其次,$\sum^k_{j=1}\theta_j^Tx$不一定为$1$。因此,我们将使用softmax函数将向量$(\theta_1,\ldots,\theta_k)$转化为每个元素都是非负的并且和为$1$的概率向量。

  定义softmax函数$\rm{softmax}:\mathbb{R}^k \rightarrow \mathbb{R}^k$为

  \begin{equation}
    \rm{softmax}(t_1,\ldots,t_k)=
    \begin{bmatrix}
      \frac{{\rm{exp}} (t_1)}{\sum^k_{j=1}{\rm{exp}} (t_j)} \\
      \vdots \\
      \frac{{\rm{exp}} (t_k)}{\sum^k_{j=1}{\rm{exp}} (t_j)} \\
    \end{bmatrix}
  \end{equation}

  softmax函数的输入,向量$t$一般被称为logits,在定义中,softmax函数的输出必为每个元素都是非负的并且和为$1$的概率向量。

  令$(t_1,\ldots,t_k)=(\theta_1^Tx,\ldots,\theta_k^Tx)$,我们将$(t_1,\ldots,t_k)$作为softmax函数的输入,将softmax函数的输出作为概率$P(y=1 | x;\theta),\ldots,P(y=k | x;\theta)$,我们得到如下概率模型:

  \begin{equation}
    \begin{bmatrix}
      P(y=1 | x;\theta) \\
      \vdots \\
      P(y=k | x;\theta) \\
    \end{bmatrix}
    =
    \rm{softmax}(t_1,\ldots,t_k)
    =
    \begin{bmatrix}
      \frac{{\rm{exp}} (\theta_1^T x)}{\sum^k_{j=1}{\rm{exp}} (\theta_j^T x)} \\
      \vdots \\
      \frac{{\rm{exp}} (\theta_k^T x)}{\sum^k_{j=1}{\rm{exp}} (\theta_j^T x)} \\
    \end{bmatrix}
  \end{equation}

  为了表示方便,我们令$\phi_i=\frac{{\rm{exp}} (\theta_i^T x)}{\sum^k_{j=1}{\rm{exp}} (\theta_j^T x)}$,上面等式可以简写为:

  \begin{equation}
    P(y=i|x;\theta) = \phi_i= \frac{{\rm{exp}} (t_i)}{\sum^k_{j=1}{\rm{exp}} (t_j)} = \frac{{\rm{exp}} (\theta_i^T x)}{\sum^k_{j=1}{\rm{exp}} (\theta_j^T x)}
  \end{equation}

  接下来,我们计算一个样例$(x,y)$的负对数-似然(log-likehood)。

  \begin{equation}
    -\log P(y|x,\theta)=-\log \left( \frac{{\rm{exp}} (t_y)}{\sum^k_{j=1}{\rm{exp}} (t_j)} \right) = -\log \left( \frac{{\rm{exp}} (\theta_y^T x)}{\sum^k_{j=1}{\rm{exp}} (\theta_j^T x)} \right)
  \end{equation}

  因此,训练数据的负对数-似然,即损失函数可以写为:

  \begin{equation}
    \mathcal{l}(\theta) = \sum^n_{i=1} - \log \left( \frac{{\rm{exp}} (\theta_{y^{(i)}}^T x^{(i)}}{\sum^k_{j=1}{\rm{exp}} (\theta_j^T x^{(i)})} \right)
  \end{equation}

\end{CJK}
\end{document}
