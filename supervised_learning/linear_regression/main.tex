\chapter{线性回归}

为了让我们的住房案例更有趣,让我们考虑一个稍微复杂些的数据集,我们额外知晓每套住房的卧室数量:

\begin{center}
  \begin{tabular}{c|c|c}
    居住面积(平方英尺) & \#卧室数量 & 价格(1000美元) \\
    \hline
    2104                 & 3          & 400              \\
    1600                 & 3          & 330              \\
    2400                 & 3          & 369              \\
    1416                 & 2          & 232              \\
    3000                 & 4          & 540              \\
    $\vdots$             & $\vdots$   & $\vdots$         \\
  \end{tabular}
\end{center}

其中,$x$为属于$\mathbb{R}^2$的二维向量。例如,$x_1^{(i)}$为训练集中第$i$套住房的居住面积,$x_2^{(i)}$为其卧室数量。(通常,当设计一个学习问题时,需要由你自己来决定选择哪些特征,因此,如果你在波特兰(Portland)收集住房数据,可能也会选择其他特征,比如,每套住房是否有壁炉及浴室的数量等。我们后续会讨论更多有关于特征选择的内容,但目前只考虑上面给出的特征。)

为了开展监督学习,我们必须决定如何在计算机中表示函数或假设$h$。作为初始选择,我们将$y$近似为一个关于$x$的线性函数:
$$
  h_\theta(x)=\theta_0+\theta_1 x_1 + \theta_2 x_2
$$
其中,$\theta_i$为参数(也称为权重),参数化从$\mathscr{X}$映射到$\mathcal{Y}$的线性函数空间。我们将$h_\theta(x)$简写为$h(x)$。为了简化我们的表示,我们引入$x_0=1$(截距项,Intercept Term),可以得到如下等式:
$$
h(x)=\sum^d_{i=0}\theta_i x_i = \theta^T x
$$
其中,我们可以将上述等式右侧的$\theta$与$x$视为向量,$d$为输入变量的数量(不计算$x_0$)。

% LMS 算法
\section{LMS算法}

我们需要找到一个$\theta$使$J(\theta)$最小化。让我们使用一种搜索算法,该算法以一个对$\theta$的初始猜测值开始,然后不断调整$\theta$使$J(\theta)$更小,直至收敛至某一个能够使$J(\theta)$最小化的$\theta$。特别地,让我们考虑梯度下降(Gradient Descent)算法,以某个初始值$\theta$开始,不断进行如下更新:
$$
  \theta_j:=\theta_j-\alpha\frac{\partial J(\theta)}{\partial \theta_j}
$$
(同时对所有$j=0,\ldots,d$使用此更新。)其中,$\alpha$被称为学习率,这是一个非常自然的算法,每次向$J$最陡峭的衰减方向前进一步。


